\addcontentsline{toc}{chapter}{Summary}

\markleft{}
\markright{}


\begingroup
\let\clearpage\relax
\let\cleardoublepage\relax
\let\cleardoublepage\relax




\selectlanguage{english}
\chapter*{Correctness of services and their composition\vspace{0.1em}\newline Summary}

\lettrine[findent=.1em,lines=2,nindent=0em]{S}{ervice-oriented} computing (\acronym{SOC}) is an emerging paradigm of system design and aims at replacing complex monolithic systems by a composition of interacting systems, called \emph{services}. A service encapsulates self-contained functionality and offers it over a well-defined, standardized interface.

This modularization may reduce both complexity and cost. At the same time, new challenges arise with the distributed execution of services in dynamic compositions. In particular, the \emph{correctness} of a service composition depends not only on the local correctness of each participating service, but also on the correct interaction between them. Unlike in a centralized monolithic system, services may change and are not completely controlled by a single party.

We study correctness of services and their composition and investigate how the design of correct service compositions can be systematically supported. We thereby focus on the communication protocol of the service and approach these questions using formal methods and make contributions to three scenarios of \acronym{SOC}.

The correctness of a service composition depends on the correctness of the participating services. To this end, we (1) study correctness criteria which can be expressed and checked with respect to a single service. We validate services against behavioral specifications and verify their satisfaction in any possible service composition. In case a service is incorrect, we provide diagnostic information to locate and fix the error.

In case every participating service of a service composition is correct, their interaction can still introduce problems. We (2) automatically verify correctness of service compositions. We further support the design phase of service compositions and present algorithms to automatically complete partially specified compositions and to fix incorrect compositions.

A service composition can also be derived from a specification, called \emph{choreography}. A choreography globally specifies the observable behavior of a composition. We (3) present an algorithm to deduce local service descriptions from the choreography which\,---\,by design\,---\,conforms to the specification.

All results have been expressed in terms of a unifying formal model. This not only allows to formally prove correctness, but also makes results independent of the specifics of concrete service description languages. Furthermore, all presented algorithms have been prototypically implemented and validated in experiments based on case studies involving industrial services.





\newpage



\selectlanguage{german}

\chapter*{Correctness of services and their composition\vspace{0.1em}\newline Zusammenfassung}

\lettrine[findent=.1em,lines=2,nindent=0em]{S}{ervice-oriented} Computing (\acronym{SOC}) ist ein Paradigma des Systementwurfes mit dem Ziel, komplexe monolithische Systeme durch eine Komposition von interagierenden Systemen zu ersetzen. Diese interagierenden Systeme werden \emph{Services} genannt und kapseln in sich abgeschlossene Funktionen, die sie \"uber eine wohldefinierte und standardisierte Schnittstelle anbieten.

Diese Modularisierung vermag Komplexit\"at und Kosten zu senken. Gleichzeitig f\"uhrt die verteilte Ausf\"uhrung von Services in dynamischen Kompositionen zu neuen Herausforderungen. Dabei spielt \emph{Korrektheit} eine zentrale Rolle, da sie nicht nur von der lokalen Korrektheit der teilnehmenden Services, sondern auch von der Interaktion zwischen den Services abh\"angt. Weiterhin k\"onnen sich Services im Gegensatz zu monolithischen Systemen ver\"andern und werden nicht von einem einzelnen Teilnehmer kontrolliert.

Wir studieren die Korrektheit von Services und Servicekompositionen und untersuchen, wie der Entwurf von korrekten Servicekompositionen systematisch unterst\"utzt werden kann. Wir legen dabei den Fokus auf das Kommunikationsprotokoll der Services. Mithilfe von formalen Methoden tragen wir zu drei Szenarien von \acronym{SOC} bei.

Die Korrektheit einer Servicekomposition h\"angt von der Korrektheit der teilnehmenden Services ab. Aus diesem Grund (1) studieren wir Korrektheitseigenschaften, die im Bezug auf einen einzelnen Service ausgedr\"uckt und \"uberpr\"uft werden k\"onnen. Wir validieren Services gegen Verhaltensspezifikationen und verifizieren ihre G\"ultigkeit in jeder m\"oglichen Servicekomposition. Falls ein Service inkorrekt ist, erarbeiten wir Diagnoseinformationen mit deren Hilfe Fehler lokalisiert und repariert werden k\"onnen.

Falls alle teilnehmenden Services einer Servicekomposition korrekt sind, kann ihre Interaktion zu Problemen f\"uhren. Wir (2) verifizieren automatisch die Korrektheit von Servicekompositionen. Weiterhin unterst\"utzen wir die Entwurfsphase von Servicekompositionen und stellen Algorithmen vor, mit denen teilweise spezifizierte Kompositionen automatisch vervollst\"andigt und mit denen inkorrekte Kompositionen automatisch korrigiert werden k\"onnen.

Einer Servicekomposition kann weiterhin von einer Spezifikation (\emph{Choreographie} genannt) abgeleitet werden. Eine Choreographie spezifiziert den Nachrichtenaustausch in einer Servicekomposition. Wir (3) erarbeiten einen Algorithmus, mit dem lokale Servicebeschreibungen aus einer Choreographie abgeleitet werden k\"onnen, die per Konstruktion der Spezifikation gen\"ugen.

Alle Resultate wurden in einem einheitlichen formalen Modell ausgedr\"uckt. Dies erm\"oglicht nicht nur formale Beweise, sondern macht die Resultate von konkreten Spezifikationssprachen unabh\"angig. Weiterhin wurden alle vorgestellten Algorithmen prototypisch implementiert und anhand von industriellen Fallstudien validiert.


\endgroup
